\frenchspacing
\documentstyle[twocolumn]{article}
\begin{document}


\title{\sc Information and Economics: \\
A Critique of Hayek}
\author{Allin F. Cottrell and W. Paul Cockshott}
\date{October, 1994}
\maketitle



\section{introduction}
\input intro
\section{hayek's argument outlined}
\input thesis
\section{hayek's subjectivism: critique}
\input subject
\section{the calculation problem and its misrepresentation}
\input calcul
\section{to centralise or not?}
\input central
\section{inadequacy of the price form}
\input price
\section{information flows under market and plan}
\input complex
\section{the argument from dynamics}
\input dynamic
\section{conclusion: evolution and history}
\input evolve

\section*{References}
\begin{description}\itemsep -.05in
\item[Althusser, L. (1971).]`Ideology and ideological state
apparatuses'. In {\it Lenin and Philosophy and Other Essays\/}.
London: NLB.  
%\item[Arrow, K. J., 1984,]{\it The Economics of Information\/}
%(Collected Papers, vol. 4), Cambridge, Mass.: Belknap. 
\item[Chaitin, G. J. (1982).]`Algorithmic information theory'. In 
{\em Encyclopedia of Statistical Sciences}, vol. 1. 
New York: Wiley, pp. 38--41.
\item[Cockshott, W. P. (1990).]`Application of artificial intelligence
techniques to economic planning'. {\it Future Computing Systems},
vol. 2, pp. 429--43.
\item[Cockshott, W. P. and Cottrell, A. (1989).]`Labour value and
socialist economic calculation'. {\it Economy and Society\/}, vol. 18,
pp. 71--99.  
\item[Cockshott, W. P. and Cottrell, A. (1993).]{\it
Towards a New Socialism}. Nottingham: Spokesman.
\item[Cottrell, A. (1994).]`Hayek's early cycle theory re-examined'.
{\it Cambridge Journal of Economics\/}, vol. 18, 197--212.  
\item[Cottrell, A. and Cockshott, W. P. (1993a).]`Calculation,
complexity and planning: the socialist calculation debate
once again'. {\it Review of Political Economy\/}, vol. 5, pp. 73--112.
\item[Cottrell, A. and Cockshott, W. P. (1993b).]`Socialist planning
after the collapse of the Soviet Union'. {\it Revue Europ\'{e}ene des
Sciences Sociales\/}, vol. 31, pp. 167--185.  
\item[Dawkins, R. (1982).]{\it The Extended Phenotype\/}. Oxford: 
Oxford University Press. 
\item[Dennett, D. C. (1991).]{\it Consciousness Explained\/}. Boston:
Little, Brown.  
\item[Dorfman, R.,] Samuelson, P. A., and Solow, R. M. (1958).
{\it Linear Programming and Economic Analysis\/}. New York: McGraw Hill.
\item[Gibbs, W. W. (1994).]`Software's chronic crisis'. {\it Scientific
American\/}, vol. 271, pp. 86--95. 
\item[Hayek, F. A. (1935).]{\it Prices and Production\/}, revised edition. 
London: Routledge.  
\item[Hayek, F. A. (1945).]`The use of knowledge in society'. 
{\it American Economic Review\/}, vol. 35, pp. 519--30. 
\item[Hayek, F. A. (1955).]{\it The Counter-Revolution of Science\/}.
New York: The Free Press.
\item[Keynes, J. M. (1936).]{\it The General Theory of Employment, Interest
and Money\/}. London: Macmillan.  
\item[Lavoie, D. (1985).]{\it Rivalry and Central Planning: The Socialist 
Calculation Debate Reconsidered\/}. Cambridge: Cambridge University Press.  
\item[Lawlor, M.S. and Horn, B.L. (1992).]`Notes on the Sraffa--Hayek 
Exchange', {\it Review of Political Economy\/}, vol 4.
\item[Lawson, T. (1992).]`Realism and Hayek: a case of continuous 
transformation', mimeo, University of Cambridge.
\item[Mises, L. (1949).]{\it Human Action: A Treatise on Economics\/}.
New Haven: Yale University Press.  
\item[Nove, A. (1977).]{\it The Soviet Economic System\/}.
London: George Allen and Unwin. 
\item[Nove, A. (1983).]{\it The Economics of Feasible Socialism\/}.
London: George Allen and Unwin.
\item[Shannon, C. E., and Weaver, W. (1949).]{\it The Mathematical Theory 
of Communication\/}. Urbana, Ill.: University of Illinois.
\end{description}
\end{document}
