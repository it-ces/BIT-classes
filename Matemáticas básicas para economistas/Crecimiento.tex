\documentclass[12pt]{article}
\usepackage{pgf,tikz}
\usepackage{graphicx}
\usepackage{amssymb,fancyhdr,txfonts,pxfonts}
\usepackage{mathrsfs}
\usepackage{amsmath}
\usetikzlibrary{arrows}
\pagestyle{empty}
\usepackage[utf8]{inputenc} 
\usepackage{enumerate}	

\providecommand{\abs}[1]{\lvert#1\rvert}
\title{Matemáticas básicas en el análisis económico}
\author{Iván Andrés Trujillo Abella}
\date{}


\maketitle

\begin{document}



\part*{Matemáticas básicas en el análisis económico}


El siguiente documento tiene como fin brindar al lector una introducción a las matemáticas básicas en un contexto aplicado del análisis económico; se exponen los conceptos que usualmente se  trabajarán durante el desarrollo (básico) de la carrera, temas que se ven  en asignaturas como  economía internacional, equilibrio económico, política económica entre otras. Este  documento no  tiene como fin sustituir los libros guía de tales asignaturas, ni mucho menos los de  la formación en matemáticas, siempre es preferible leer y estudiar los libros que se  consideren adecuados dentro de los planes de estudio, que son en su mayoría las referencias utilizadas para este trabajo. El estudiante podrá consultar en este trabajo  conceptos elementales dado que este texto tiene como objetivo que se afiancen los conceptos matemáticos con ayuda de los módulos de microeconomía y crecimiento económico,  dado que estas dos disciplinas son transversales al denominado análisis económico también servirá como apuntes o notas de clase  para estas asignaturas.

Se ha tratado en lo posible dejar los conceptos de manera general para que el estudiante empiece a reconocer de manera global como trabajar con el análisis abstracto, sin tener que recurrir al método numérico directamente, aunque por supuesto se tratará de desarrollar ejemplos o por lo menos se propondrán  ejercicios con este fin, la exposición de la matemática que los autores han utilizado corresponde por supuesto a una metodología más que formal intuitiva y práctica por la convicción de que será fundamental este enfoque en el desempeño de las asignaturas del programa académico.

\newpage

\subsection{funciones}

Este capitulo introductorio será esencialmente un repaso breve, del concepto de función, y se trabajará un poco con las funciones mas utilizadas en el análisis económico introductorio; como la función lineal, cuadrática, hipérbola rectangular, exponencial y logarítmica. 

\\

\textbf{Función};

Una  función es una regla de asignación ($f$) específica, que relaciona los elementos  de un conjunto $A$ con los elementos de un conjunto $B$ de tal manera que a cada elemento del conjunto  $A$ le pertenece ( o se le asocia) uno y solo uno de los elementos del conjunto $B$.

Expresando tal relación de la siguiente manera;

$$ f: A \longrightarrow B $$

entonces, la función es una regla determinada que le asigna a cada elemento A un solo elemento del conjunto B, pero un elemento del conjunto B puede tener varios elementos asociados de A. Por ejemplo la función $x^{2}$ que podemos decir que genera las parejas ordenadas de la siguiente manera  $(x,y)$ y $(-x,y)$ dado que $x\neq-x$ entonces esta es una función.

Esta definición la podemos ampliar desde la definición de pares ordenados que se dará enseguida.

Una función como $f(x)=y$ se conoce como la función identidad,  note que cumple la definición que hemos adoptado de función puesto que a cada  argumento $x$ le corresponde una y solo una $y$, no es función $\sqrt{x}$, puesto que para cada $x$  $y$ puede tomar dos valores $(-y)$ y $y$. Lo que no permite establecer una relación única, note que esto implicó que la primera componente de la pareja ordenada $(x,y)$  no puede aparecer en dos pares ordenados a menos que la segunda componente sea igual para todos los casos. De lo que se deduce que; dados los conjuntos $A$ y $B$ tenemos que una función se define como todos los pares de la forma $(a,b)$ donde $a$ pertenece al conjunto  $A$ y $b$ pertenece al conjunto $B$ donde únicamente puede aparecer $(a,b)$ y $(a,c)$ si y solo sí $b=c$.

Podemos definir las siguientes reglas de asignación por ejemplo;

$$ \Theta: \Re \longrightarrow \Re$$

o el concepto de probabilidad clásico, de eventos equiprobables.

$$ \chi: \Re \longrightarrow [0,1]$$

son ejemplos estas  de funciones. La ultima función, es muy útil para la aplicación de problemas del azar, se puede definir como $f(x)=1/n$ donde n es el número de eventos equiprobables.

puede el lector determinar explicitamente una función de la forma;

$$f:\Re \longrightarrow \Re^{+}$$

Esta función es muy común en microeconomía en la teoría del consumidor. el lector debería responder si la siguiente expresión cumple con la definición de función; $x^{2}+y^{2}=r^{2}$.

\subsubsection{Función lineal}

La mayor parte de los modelos económicos introductorios se trabajan con las funciones lineales, por eso es indispensable que se conozca esta función.

Antes es necesario hablar de un concepto que necesitamos desarrollar el de \textbf{pendiente}; este concepto se debe afianzar puesto que es de vital  importancia, la pendiente $m$ (como usualmente se denota en los textos) está definida como 

$$ m = \dfrac{f(x_{2})-f(x_{1})}{x_{2}-x_{1}}$$


en otras palabras la pendiente es el cambio de posición en $\Delta f(x)$ con respecto a $\Delta x$, para tratar de entender el concepto de pendiente se usará el caso particular de ecuación fundamental en el movimiento rectilíneo uniformemente acelerado, de la ya conocida ecuación de la velocidad $v=\dfrac{d}{t}$ de sus clases de física, donde $v$ es la velocidad, $d$ la distancia recorrida y $t$ es el tiempo transcurrido, esto nos indica que por ejemplo un objeto se desplaza cada $t$ unidades de tiempo $y$  unidades de distancia.


la relación con la función lineal es indispensable debido a que esta función tiene una función constante para todo el dominio, es decir que los cambios en la variable independiente (y) es constante o cambia en la misma proporción cuando (x) aumenta.


la función lineal toma la forma;

$$y=\theta x+ b$$

es decir que la pendiente $\theta$ me indica cuantas unidades varía o aumenta $y$ cuando $x$	 aumenta.  y la intersección con el eje y esta dado por $f(0)$.

\textit{dos rectas son paralelas}; las dos funciones que se presentan en la figura tal, son paralelas es decir que nunca se interceptan por que ambas pendientes son iguales es decir lo único que varía es el punto de intersección con el eje y. note que $c>b$,  y que específicamente $c>0$ puesto que esta pasa por el origen. En [x] se presenta una sencilla demostración  del teorema de las paralelas.

\begin{theorem}
Dos rectas son paralelas si ambas tiene la misma pendiente

\end{theorem}

\begin{eqnarray}
 
\begin{center}

y_{1} = y_{2}

\,

a_{1}x + b_{1} = a_{2}x +b_{2}


\,

x(a_{1} - a_{2})=b_{2} - b_{1}

\,


x = \dfrac{( b_{2} - b_{1})}{(a_{1} - a_{2})}

\therefore a_{1} \neq a_{2}



\end{center}


\end{eqnarray}

de esta manera se demuestra que las rectas son paralelas si ambas tienen la misma pendiente.


\begin{figure}

\begin{center}

\definecolor{qqwuqq}{rgb}{0.12941176470588237,0.12941176470588237,0.12941176470588237}
\begin{tikzpicture}[line cap=round,line join=round,>=triangle 45,x=0.5cm,y=0.5cm]
\draw[->,color=black] (-6.,0.) -- (6.,0.);
\foreach \x in {-6.,-5.,-4.,-3.,-2.,-1.,1.,2.,3.,4.,5.}
\draw[shift={(\x,0)},color=black] (0pt,-2pt);
\draw[->,color=black] (0.,-9.) -- (0.,9.);
\foreach \y in {-9.,-8.,-7.,-6.,-5.,-4.,-3.,-2.,-1.,1.,2.,3.,4.,5.,6.,7.,8.}
\draw[shift={(0,\y)},color=black] (-2pt,0pt);
\clip(-6.,-9.) rectangle (6.,9.);
\draw[line width=1.2pt,color=qqwuqq,smooth,samples=100,domain=-6.0:6.0] plot(\x,{2.0*(\x)});
\draw[line width=1.2pt,color=qqwuqq,smooth,samples=100,domain=-6.0:6.0] plot(\x,{2.0*(\x)+3.0});
\draw (-1,7.98) node[anchor=north est] {$y=ax+b$};
\draw (2,2) node[anchor=north west] {$y=ax+c$};
\end{tikzpicture}

\caption{paralelas}

\end{center}
\end{figure}






\newpage







un ejemplo donde se utiliza la función lineal lo podemos encontrar en el modelo de ventajas comparativas.

o en la restricción presupuestaria que indica por ejemplo la combinación de bienes que se pueden comprar dados unos precios y un ingreso.



consideremos el caso de como la función general de manera lineal dado un punto y la pendiente; es decir sabemos que es proporcionalidad esto quiere decir que si por el punto $a$ deben recaer puntos tales que sean proporcionales en sus cambios; 

$$\dfrac{y-y_{1}}{x-x_{1}}= m$$ podemos reescribir como; 
$$(y-y_{1})=m(x-x_{1})$$
$$y=m(x-x_{1})+ y_{1}$$




Problema; Cada tres años el valor de las casas disminuye exactamente $\beta$ \$ , si el valor promedio de las casas es de $\alpha$ \$, ¿como es la formula funcional que nos permite establecer una relación para este fenómeno económico?

otro problema es dada la información de la pendiente y  un punto de la función hallar su forma funcional; es decir problema; u agricultor utiliza un fertilizante el número de sacos de café que produce es de $5 $ por cada 1 lt de solución acuosa del fertilizante se ha tomado un dato $(3,19)$ hallar la producción del agricultor cuando utiliza 8 litros de la solución. 




El lector recordará por sus clases de geometría el muy útil y prestigioso \textbf{teorema de pitágoras} que enuncia que de un triangulo rectángulo ( que posee un angulo de 90 grados)  la hipotenusa al cuadrado es igual a la suma de los catetos al cuadrado. es decir $h^{2}= o^{2}+a^{2}$


\begin{figure}

\begin{center}

\definecolor{zzttqq}{rgb}{0.26666666666666666,0.26666666666666666,0.26666666666666666}
\begin{tikzpicture}[line cap=round,line join=round,>=triangle 45,x=1.0cm,y=1.0cm]
\clip(0.1,0.04) rectangle (5.,4.);
\fill[color=zzttqq,fill=zzttqq,fill opacity=0.10000000149011612] (0.9,3.62) -- (0.86,1.12) -- (4.62,1.06) -- cycle;
\draw [color=zzttqq] (0.9,3.62)-- (0.86,1.12);
\draw [color=zzttqq] (0.86,1.12)-- (4.62,1.06);
\draw [color=zzttqq] (4.62,1.06)-- (0.9,3.62);
\draw (1.12,0.81) node[anchor=north west] {$a ^ {2} + b^{2} = c^{2} $};
\draw (0.48,2.78) node[anchor=north west] {$a$};
\draw (2.,3.08) node[anchor=north west] {$c$};
\draw (2.06,1.18) node[anchor=north west] {$b$};
\end{tikzpicture}
\caption{pitágoras}


\end{center}

\end{figure}




pues en un sistema  de coordenadas rectangulares, este teorema sirve para enunciar la distancia entre dos puntos, de la siguiente manera; $\d(a,b)=\sqrt{((x_{i+1}-x_{i})^{2} + ((y_{i+1}-y_{i})^{2}}$



\begin{figure}

\begin{center}


\definecolor{zzttqq}{rgb}{0.26666666666666666,0.26666666666666666,0.26666666666666666}
\definecolor{qqqqff}{rgb}{0.3333333333333333,0.3333333333333333,0.3333333333333333}
\begin{tikzpicture}[line cap=round,line join=round,>=triangle 45,x=1.0cm,y=1.0cm]
\draw[->,color=black] (-1.,0.) -- (9.6,0.);
\foreach \x in {-1.,1.,2.,3.,4.,5.,6.,7.,8.,9.}
\draw[shift={(\x,0)},color=black] (0pt,-2pt);
\draw[->,color=black] (0.,-1.) -- (0.,8.);
\foreach \y in {-1.,1.,2.,3.,4.,5.,6.,7.}
\draw[shift={(0,\y)},color=black] (2pt,0pt) -- (-2pt,0pt);
\clip(-1.,-1.) rectangle (9.6,8.);
\fill[color=zzttqq,fill=zzttqq,fill opacity=0.10000000149011612] (2.14,1.74) -- (7.4,6.76) -- (7.38,1.74) -- cycle;
\draw [dash pattern=on 2pt off 2pt] (2.14,-1.) -- (2.14,8.);
\draw [dash pattern=on 2pt off 2pt] (7.4,-1.) -- (7.4,8.);
\draw [dash pattern=on 2pt off 2pt,domain=-1.:9.6] plot(\x,{(-6.76-0.*\x)/-1.});
\draw [dash pattern=on 2pt off 2pt,domain=-1.:9.6] plot(\x,{(-1.74-0.*\x)/-1.});
\draw [color=zzttqq] (2.14,1.74)-- (7.4,6.76);
\draw [color=zzttqq] (7.4,6.76)-- (7.38,1.74);
\draw [color=zzttqq] (7.38,1.74)-- (2.14,1.74);
\draw (2.2,0.16) node[anchor=north west] {$x_{i}$};
\draw (6.66,0.06) node[anchor=north west] {$x_{i+1}$};
\draw (-0.48,2.4) node[anchor=north west] {$y_{i}$};
\draw (-0.8,6.86) node[anchor=north west] {$y_{i+1}$};
\draw (4.92,2.58) node[anchor=north west] {$x_{i+1} - x_{i}$};
\draw (7.68,4.42) node[anchor=north west] {$y_{i+1} - y_{i}$};
\draw (2.33 , 6.66) node[anchor=north west] {$\sqrt{(x_{i+1} - x_{i})^{2} + (y_{i+1}-y_{i}})^{2}  $};

\end{tikzpicture}

\caption{distancia entre dos puntos}


\end{center}

\end{figure}

Es esencial que se conozca este concepto de distancia (note que no es el único), puesto que muchas de las fórmulas "institucionalizadas" en el desarrollo matemático dependen de ella, como por ejemplo el de la circunferencia.  y esta nos será útil para trabajar un concepto económico conocido como la frontera de posibilidades de producción.


veremos que $$(x-h)^{2} + (y-k)^{2}=r^{2}$$ será muy útil para introducir un concepto de costo de oportunidad.

para simplificar supongamos que nuestra circunferencia esta centrada en el punto de origen por lo que $h+k=0$. Es por lo tanto necesario que el lector observe que es la misma ecuación de la distancia entre dos puntos, mencionada anteriormente;$(x)^{2}+(y)^{2}=r^{2}$ ahora como en el análisis económico nos interesa solo cantidades positivas tenemos que; restringirlo así que hallando 
$$y=\sqrt{r^{2}-x^{2}}$$

pero necesitamos que $r^{2}-x^{2}\geq 0$ así tomaremos los $r\geq x $  no negativos.







 



\newpage


la parábola; 

$$ y= x^{2} + \varphi $$

el punto más alto se conoce como vértice y es fácil hallar $f(0)$ que da como resultado el par ordenado $(0,\varphi)$. Hemos incluido esta función para que se tenga en cuenta de su ilustración, la simetría con respecto al eje  y.

\begin{figure}

\begin{center}

\definecolor{qqwuqq}{rgb}
{0.12941176470588237,0.12941176470588237,0.12941176470588237}
\begin{tikzpicture}[line cap=round,line join=round,>=triangle 45,x=1.0cm,y=0.4cm]
\draw[->,color=black] (-4.,0.) -- (4.,0.);
\foreach \x in {-4.,-3.,-2.,-1.,1.,2.,3.}
\draw[shift={(\x,0)},color=black] (0pt,-2pt);
\draw[->,color=black] (0.,0.) -- (0.,9.);
\foreach \y in {,1.,2.,3.,4.,5.,6.,7.,8.,9.}
\draw[shift={(0,\y)},color=black] (2pt,0pt) -- (-2pt,0pt);
\clip(-4.,0.) rectangle (4.,9.);
\draw[line width=1.2pt,color=qqwuqq,smooth,samples=100,domain=-4.000000000000007:3.999999999999996] plot(\x,{(\x)^(2.0)});
\begin{scriptsize}
\draw[color=qqwuqq] (-2.62,8.17) node {$f$};
\end{scriptsize}
\end{tikzpicture}
\label{fig 1}

\caption{función tal}



\end{center}

\end{figure}


esto es importante debido a que $f(-x)=f(x)$ lo cual será muy útil para desarrollos posteriores en la matemática.



\textbf{Operaciones con funciones}

Se pueden operar funciones, como su composición, suma, resta, multiplicación y división entre otras.









\textbf{función inversa}

La función inversa es importante en el análisis económico por eso la hemos incluido, para que una  una función $f$ tenga inversa $f^{-1}$ se debe cumplir la condición $f$ de que la función es biunívoca. Es decir que si $f(x_{1}) = f(x_{{2})$ entonces, $ x_{1} = x_{2}$ 

Es decir para la parábola en todo su dominio no podemos definir una función inversa debido a que como ya se dijo es simétrica y entonces $f(-x)=f(x)$ y $-x\neq x$.

la función inversa actúa de manera especial en economía puesto que muchos de los modelos matemáticos hacen uso de su concepto, sabrá por su curso de introducción a la economía que la demanda $Q=f(p)$ con la función inversa nos aseguramos que $p=f^{-1}(Q)$, observe que antes por la regla de asignación de $f$ obteníamos de un conjunto de precios, las cantidades demandadas ahora $f^{-1}$ nos permite obtener del conjunto de demandas el precio respectivo.





\subsection{función cuadrática}

La función cuadrática es, muy útil en economía, la curva de lactancia ha sido muy estudiada para saber conocer la producción lechera; es decir el conocimiento predicción de la leche por medio de un modelo que puda ayudarnos a predecir esto de la forma;


$$y=ax+bx^{2}+c$$ así la producción esta curva nos permite conocer tales estimaciones, en la producción lechera.







\subsection*{Función exponencial y logarítmica}




La importancia de la función exponencial, radica en una variedad de problemas prácticos que se pueden resolver por ejemplo, cada día un población de bacterias crece en el doble que el día anterior pasado (n) días cuantas bacterias hay, igualmente para el caso del capital, y otros ejemplos, aquí la utilidad radica en el potencial de quien es capaz de modelar diferentes situaciones con esta, primero trataremos de analizar  un ejemplo que se expone en "que es  y que no es la estadística" 

pero antes una situación planteada con anterioridad que no es exactamente este enunciado pero sigue la misma estructura; los peces de un estanque se reproducen a diario cada día hay el doble de peces que el día anterior si a los 48 días el estanque esta al 100\% cuantos días son necesarios para que el estanque esté  a la mitad?

se sigue el siguiente razonamiento, si cada día hay el doble del día anterior entonces la población de ayer estrictamente fue la mitad de la de hoy. esto quiere decir que si hoy esta el estanque lleno ayer estuvo a la mitad es decir 47 días fueron necesarios para que esté a la mitad. Pero a (t) días que tan lleno estará el estanque? esto implica inmediatamente tratar de construir una ecuación que nos describa tal comportamiento esto es;

$Q(t)$la cantidad de bacterias en el día(t) es?

$$Q(48)=100$$ 
$$Q(47)=100(0.5)$$
$$Q(46)=100(0.5)^{2}$$
$$...$$
$$Q(t)=100(0.5)^{48-t}$$
que era la ecuación que estábamos buscando para el problema, en general cuando t es cualquier número ahora supongamos que tenemos el problema de que conocemos el número de peces porcentaje esta lleno de l estanque pero no sabemos cuantos días han pasado, entonces debemos saber acerca de los logaritmos aquí viene a jugar un papel importante la base de la potenciación.


entonces supongamos que tenemos dos número reales tales que;



Por lo que de manera general conocer el problema queda resuelto a unos parámetros establecido según estemos interesados. Por lo tanto este mismo problema se podrá modelar siguiendo una estructura más general.


$$l(t)=N_{0} \theta^{\gamma-t}$$

\subsubsection{Logaritmos} 
 
 
 

La introducción de los logaritmos se hace indispensable en un texto para economistas además, esta ingeniosa estructura matemática tuvo un valor histórico muy significativo como se relata en el texto de ian stewart. El uso de los logaritmos en economía es importante, tanto para el análisis empírico como teórico.

Los logaritmos fueron introducidos como un método 




La base de los logaritmos neperianos, es el número $e$, del cual hablaremos más adelante por ahora diremos que para una base cualquiera $a$ el logaritmo  de un número positivo en esa base se es


$$\log_{a}(Y)=x$$

$$a^{x}=Y$$

por lo tanto un logaritmo es un exponente, y como consecuencia directa  sus propiedades están en intima relación con las reglas de la potenciación que el lector ha adquirido en sus cursos de matemáticas más elemental. 

así $H=a^{x}$ y $L=a^{y}$ entonces;

$$HL= a^{x+y}$$
$$\dfrac{H}{L}= a^{x-y}$$
$$ H^{n}= (a^{x})^{n}= a^{xn}$$





Así es sencillo deducir las Propiedades de los logaritmos.



$$\log(HL)= \log x + \log y$$ 




$$ \log(\dfrac{H}{L})= \log x - \log y$$

Sin embargo, los logaritmos en el análisis económico tienen un base preferida como dijimos anteriormente el número $e$, 

$$ \ln x = \log_{e}x$$

\subsubsection{El número e}

Este número es de radical importancia en matemáticas, es número irracional  y completamente intrigantes por sus propiedades.
 



\section{cálculo diferencial}

Como se verá el concepto de derivada es fundamental debido a que las relaciones entre magnitudes no son siempre constantes,  ¿podría explicar el lector con un ejemplo a que se debe de que la pendiente como la hemos definido es insuficiente para tales relaciones? 


 

\subsection*{Optimización}

En esta sección solo daremos una breve introducción intuitiva debido a que la teoría de la optimización es de suma importancia se recomienda leer directamente este tema en cualquiera de las referencias.


En la figura que sigue se representa la función $F(x)$ para un intervalo de su dominio como se puede evidenciar, existen dos puntos en los cuales $\dfrac{dF(x)}{dx}=f(x)$ toma valor cero. Esto se evidencia en los puntos $x_{0} $ y $ x_{1}$.


\begin{figure}

\begin{center}

\definecolor{xdxdff}{rgb}{0.6588235294117647,0.6588235294117647,0.6588235294117647}
\definecolor{qqqqff}{rgb}{0.3333333333333333,0.3333333333333333,0.3333333333333333}
\definecolor{qqwuqq}{rgb}{0.12941176470588237,0.12941176470588237,0.12941176470588237}
\begin{tikzpicture}[line cap=round,line join=round,>=triangle 45,x=1.0cm,y=1.0cm]
\draw[->,color=black] (0.,0.) -- (6.,0.);
\foreach \x in {,1.,2.,3.,4.,5.}
\draw[shift={(\x,0)},color=black] (0pt,-2pt);
\draw[->,color=black] (0.,0.) -- (0.,4.);
\foreach \y in {,1.,2.,3.}
\draw[shift={(0,\y)},color=black] (2pt,0pt) -- (-2pt,0pt);
\clip(0.,0.) rectangle (6.,4.);
\draw[line width=1.2pt,color=qqwuqq,smooth,samples=100,domain=0.0:6.0] plot(\x,{sin(((\x))*180/pi)+2.0});
\draw (0.8,4.12) node[anchor=north west] {$f\textasciiacute\textasciiacute(x_o)=0$};
\draw [dash pattern=on 2pt off 2pt] (1.54,3.06)-- (1.54,0.);
\draw (1.82,0.64) node[anchor=north west] {$x_{0}$};
\draw (4.06,2.18) node[anchor=north west] {$f\textasciiacute\textasciiacute(x_1)=0$};
\draw [dash pattern=on 2pt off 2pt] (4.64,1.0026189383019068)-- (4.64,0.);
\draw (5.,0.64) node[anchor=north west] {$x_{1}$};
\begin{scriptsize}
\draw[color=qqwuqq] (-4.1,2.69) node {$f$};
\draw [fill=qqqqff] (1.54,3.06) circle (2.5pt);
\draw[color=qqqqff] (1.68,3.43) node {$B$};
\draw [fill=xdxdff] (1.54,0.) circle (2.5pt);
\draw[color=black] (1.28,1.71) node {$g$};
\draw [fill=xdxdff] (4.64,1.0026189383019068) circle (2.5pt);
\draw[color=xdxdff] (4.78,1.37) node {$F$};
\draw [fill=xdxdff] (4.64,0.) circle (2.5pt);
\draw[color=black] (4.38,0.69) node {$h$};
\end{scriptsize}
\end{tikzpicture}
\caption{optimization}

\end{center}


\end{figure}

Esto quiere decir que encontrar un punto de la función para la cual su derivada en ese punto es igual a cero no es una condición suficiente para establecer el máximo global de tal función. Sin embargo, usualmente en economía este problema queda descartado por el tipo de función con la cual se trabaja donde podemos estar seguros que ese punto donde la derivada es nula es el máximo.



\subsubsection{Multiplicador de Lagrange}

Para los óptimos restringidos para los cuales el problema puede tornarse mucho mas complicado puesto que existe una limitación en los valores que $X$ puede tomar, esto debe ser incorporado en el criterio del máximo.

El multiplicador de Lagrange es un método que nos ayudará a resolver este problema, para comprender intuitivamente como trabaja supongamos que debemos invertir en dos compañías y que las tasas de retorno de ambas 
son $r_{1}$ y$ r_{2}$ lo interesante del hecho es que desde esta perspectiva ya hemos resuelto la mitad del problema puesto que ya conocemos cual es la retribución por cada peso invertido, ahora debemos saber que tanto debemos invertir en cada una de ellas, se debe escoger la cantidad de dinero que nos retorne por las dos es la misma cantidad por que si no fuera así no se estaría maximizando los recursos pues una me ofrecerá mayor retorno sobre las otras, en otras "palabras";

siendo $\pi_{1}$ , $\pi_{2}$ la ganancia de invertir en la primera y segunda empresa, y $k_{1}$ y $k_{2}$ es la cantidad de dinero invertida en las empresas respectivamente y $k_{1}+ k_{2} =K $ el total disponible de capital. Entonces la condición para optimizar mis beneficios es 

$$\dfrac{\pi_{1}}{k_{1}}=\dfrac{\pi_{2}}{k_{2}}$$

Así decimos que la ganancia por peso invertido debe ser igual para las dos compañías.

De manera general tenemos una función objetivo $Z=f(x_1,x_2,x_3,...,x_n)$ que queremos maximizar, es decir tener las cantidades de los $x_{i}$ que me permitan obtener el valor más grande de  $Z$ sujeto a una restricción presupuestaria de la forma $K=\sum_{1}^{n}p_{i}x_{i}$.

$$\ell=f(x_1,x_2,x_3,...,x_n) + \lambda(K-\sum_{1}^{n} p_{i}x_{i})$$

obteniendo las condiciones necesarias;

\

$$\dfrac{\partial \ell }{\partial x_{i}} = \dfrac{\partial f(x_1,x_2,x_3,...,x_n)}{\partial x_{i}}+\lambda p_{i}= 0$$

esto es para todos los $({i})$

$$\dfrac{\partial \ell }{\partial x_{1}} = \dfrac{\partial f(x_1,x_2,x_3,...,x_n)}{\partial x_{1}}+\lambda p_{1}= 0$$

$$\dfrac{\partial \ell }{\partial x_{2}} = \dfrac{\partial f(x_1,x_2,x_3,...,x_n)}{\partial x_{2}}+\lambda p_{2}= 0$$


$$....$$
$$....$$

$$\dfrac{\partial \ell }{\partial x_{n-1}} = \dfrac{\partial f(x_1,x_2,x_3,...,x_n)}{\partial x_{n-1}}+\lambda p_{n-1}= 0$$

$$\dfrac{\partial \ell }{\partial x_{n}} = \dfrac{\partial f(x_1,x_2,x_3,...,x_n)}{\partial x_{n}}+\lambda p_{n}= 0$$

por ultimo 

$$\dfrac{\partial \ell}{\partial \lambda} = K-\sum_{1}^{n} p_{i}x_{i} = 0$$




Como se puede observar este sistema requiere inmediatamente que; 

$$ \lambda = \dfrac{\dfrac{\partial f(x_1,x_2,x_3,...,x_n)}{\partial x{i}}}{p_{i}} =  \dfrac{\dfrac{\partial f(x_1,x_2,x_3,...,x_n)}{\partial x{i+1}}}{p_{i+1}} = ... = \dfrac{\dfrac{\partial f(x_1,x_2,x_3,...,x_n)}{\partial x{n}}}{p_{n}}= \lambda $$

esta es la idea básica introduciendo el multiplicador $(\lambda)$.

\
\

\textbf{Ejemplo}; 

supongamos que se tiene la siguiente función $ Z=x_{1}^{\alpha}x_{2}^{\beta}$ restringida al caso de $I=p_{1}x_{1}+p_{2}x_{2}$ así con la conclusión establecida, no habrá necesidad de plantear otra ves el problema puesto que el lector ha visto que acabamos de desarrollar un modelo general.

entonces; 

$$\dfrac{\partial Z}{\partial x_{1}}=\alpha x_{1}^{\alpha -1} x_{2}^{2}$$

$$\dfrac{\partial Z}{\partial x_{2}}=\beta x_{2}^{\beta -1}x_{1}^{\alpha}$$


tenemos que 

$$\dfrac{\alpha x_{1}^{\alpha -1} x_{2}^{\beta}}{p_{1}}=  \dfrac{\beta x_{2}^{\beta -1}x_{1}^{\alpha}}{p_{2}}$$

ahora si obtenemos las cantidades de $x_{1}$ y $x_{2}$ que maximizan las cantidades.

primero debemos despejar ya sea $x_{1}$ o $x_{2}$ y después llevarlo a la restricción presupuestaria esto es;


$$ \dfrac{ \alpha x_{1}^{-1}}{p_{1}} = \dfrac{\beta x_{2}^{-1}}{p_{2}}$$


$$x_{1}= \dfrac{\alpha p_{2} x_{2}}{\beta p_{1}}$$

[*]

llevando a la restricción presupuestaria;

$$ p_{1} \left( \dfrac{\alpha p_{2} x_{2}}{\beta p_{1}} \right) + p_{2} x_{2} $$

$$ \dfrac{ \alpha x_{2} p{2} }{\beta} + p_{2}x_{2}= I$$

$$x_{2}p_{2}(\dfrac{\alpha}{\beta}+1) = I$$

$$x_{2}=\dfrac{I}{p_{2}  \left(   \dfrac{\alpha + \beta }{\beta} \right)}$$

ahora este resultado lo llevamos a $x_{1}$  a [*]

así obtenemos;

$$ x_{1} = \dfrac{\alpha p_{2} \left( \dfrac {I} {p_{2} \left( \dfrac{\alpha + \beta} {\beta} \right)} \right)} {\beta p_{1}} $$

$$x_{1}= \dfrac{\alpha I }{\beta p_{1} \left( \dfrac{ \alpha + \beta}{\beta} \right) } $$


$$ x_{1} =  \dfrac{\alpha I}{P_{1}(\alpha + \beta)}$$


para comprobar que nuestra relación es cierta ahora necesitamos  saber que realmente $p_{1}x_{1}+p_{2}x_{2}=I$.


$$p_{1} \left( \dfrac{\alpha I}{p_{1} (\alpha + \beta) } \right) + p_{2} \left(  \dfrac{I}{p_{2} \left( \dfrac{\alpha + \beta}{\beta} \right)} \right )= I $$






 $$ = \dfrac{I}{ p_{2} \left( \dfrac{\alpha + \beta}{\beta} \right)} + \dfrac{\alpha I}{\alpha + \beta} $$ = 


$$\dfrac{ \beta I}{\alpha + \beta } + \dfrac{\alpha I}{\alpha + \beta} =I \dfrac{(\alpha + \beta)}{(\alpha + \beta)}= I$$

esta función que acabamos de maximizar es muy utilizada en el análisis microeconómico y es nuestra puerta de entrada a este modulo, antes hablaremos de la importancia de la función. Esta función tiene dos interpretaciones;  se puede interpretar desde el punto de vista del consumidor o de la firma. En el primer caso la función que tratamos de maximizar representa la utilidad de un individuo de  consumir $x_{1}$ y $x_{2}$ y en el segundo tratamos de maximizar los beneficios económicos en el proceso productivo.

Entonces si observamos el gasto que realiza el consumidor en el bien $x_{1}$  que maximiza su utilidad es 
 $$ p_{1} x_{1} =  \dfrac{\alpha I}{(\alpha + \beta)}$$ 
 
  suponiendo que $(\alpha + \beta)=1$ observamos que se gasta una parte constante de la renta  $( \alpha I)$ y es por esta razón que hacemos este supuesto pues los gastos del consumidor en el bien $x_{1}$ es una parte constante de la renta y debe disponer del $100\%$ de esta. Igualmente para el caso del bien dos  
  
  $$p_{2} x_{2} = I\beta}$$
  

Un análisis análogo de la participación de los factores productivos en el producto  se da para el caso de la firma el lector debe ser capa de dar una explicación intuitiva.

Examinemos el caso concreto cuando se consumen tres bienes donde el consumidor asigna una parte constante de su renta o ingreso para cada una de las cantidades óptimas, pero antes tendremos en cuenta que la función de producción Cobb-Douglas es simplemente el caso especial cuando $i=2$ de la siguiente función;

$$\prod_{i=1}^n x_i^{\alpha_i}  $$ [1]


$$ \dfrac {\partial \prod_{i}^{n} x_{i}^{\alpha_{i}}}{\partial x_{i}}= (x_i)^{-1}\alpha_i\prod_{i=1}^{n}x_{i}^{\alpha_i}$$

y por lo tanto podemos plantear el lagrangeano


$$\ell = \prod_{i=1}^{n} x_{i}^{\alpha} - \lambda (I- \sum_{i=1}^{n}p_{i}x_{i} )$$

$$\dfrac{\parial \ell}{\partial x_{i}} = \alpha_{i}x_{i}^{-1} \prod_{i=1}^{n} x_{i}^{\alpha_{i}} - \lambda p_{i}= 0$$

como es para todos los $i$ tendríamos un sistema de ecuaciones por ahora basta establecer que 

$$\lambda=   \dfrac{\alpha_{i} x_{i}^{-1} \prod_{i=1}^{n} x_{i} ^{\alpha_{i}}}{p_{i}} = \dfrac{\alpha_{i+1} x_{i+1}^{-1} \prod_{i=1}^{n} x_{i} ^{\alpha_{i}}}{p_{i+1}}  =   ... \dfrac{\alpha_{i+k} x_{i+k}^{-1} \prod_{i=1}^{n} x_{i} ^{\alpha_{i}}}{p_{i+k}}  \lambda  $$


Escojamos $x_{i}$ y $x_{i+1}$

$$\dfrac{\alpha_{i} x_{i}^{-1} \prod_{i=1}^{n} x_{i} ^{\alpha_{i}}}{p_{i}} = \dfrac{\alpha_{i+1} x_{i+1}^{-1} \prod_{i=1}^{n} x_{i} ^{\alpha_{i}}}{p_{i+1}} = \dfrac{\alpha_{i}}{x_{i}p_{i}} = \dfrac{\alpha_{i+1}}{x_{i+1}p_{i+1}}$$ 


despejando tenemos que 

$$x_{i}=\dfrac{x_{i+1} p_{i+1} \alpha_{i}}{\alpha_{i+1} p_{i} }$$ [**]

llevando a la restricción presupuestaria tenemos que; 

$$p_{i} \left(\dfrac{x_{i+1} p_{i+1} \alpha_{i}}{\alpha_{i+1} p_{i} } \right)+ p_{i+1} x_{i+1}+ p_{i+2}x_{i+2} .... p_{n}x_{n} = I$$


$$p_{i+1} x_{i+1} \left( \dfrac{\alpha_{i}}{\alpha_{i+1}} +1 \right) + p_{i+2}x_{i+2} .... p_{n}x_{n} = I$$

$$ p_{i+1} x_{i+1} \left( \dfrac{\alpha_{i}}{\alpha_{i+1}} +1 \right) = I - \sum_{i=1}^{n}p_{i+2}x_{i+2}$$

$$x_{i+1}=\dfrac{I - \sum_{i=1}^{n}p_{i+2}x_{i+2}}{p_{i+1}  \left(  \dfrac{\alpha_{i}}{\alpha_{i+1}} +1 \right) }$$

realizando las operaciones algebraicas ( el lector debería comprobarlo); 

$$x_{i}= \dfrac{(I- \sum_{i=1}^{n}p_{i+2}x_{i+2}) \alpha_{i}}{p_{i}(\alpha_{i}+ \alpha_{i+1})}$$

ahora si procedamos con el modelo de tres variables que sería un caso particular. 

\textbf{ejemplo};


$$ Max  x_{1}^{\alpha_{1}} x_{2}^{\alpha_{2}} x_{3}^{\alpha_{3}}$$

$$sj p_{1}x_{1}+p_{2}x_{2}+p_{3}x_{3}=I$$

reemplazando en las ecuaciones correspondientes anteriores $x_{i}$ y $x_{i+1}$ tenemos que;



$$x_{1}= \dfrac{(I-p_{3}x_{3})\alpha_{1}}{p_{1}(\alpha_{1} + \alpha_{2})}$$

$$x_{2}= \dfrac{I - p_{3} x_{3}}{p_{2}  (\dfrac{ \alpha_{1}}{\alpha_{2}} +1) } $$

ahora reemplazamos  en [**] por $i=3$ obteniendo;

$$x_{3}= \dfrac{p_{1}x_{1}\alpha_{3}}{\alpha_{1} p_{3}}$$

ahora reemplazando $x_{1}$ en esta ecuación obtenemos las cantidades de $x_{3}$ que maximizan mi utilidad así que; 

\begin{eqnarray}

x_{3}= \dfrac{\alpha_{3}I - p_{3}x_{3} \alpha_{3}}{p_{3} (\alpha_{1} + \alpha_{2}}

\dfrac{ x_{3}+x_{3}\alpha_{3}}{\alpha_{1} + \alpha _{2}} = \dfrac{\alpha_{3}I}{p_{3}(\alpha_{1}+\alpha_{2})}

x_{3} (\dfrac{\alpha_{1}+\alpha_{2}+\alpha_{3}}{\alpha_{1}+\alpha_{2}}) = \dfrac{\alpha_{3}I}{p_{3}(\alpha_{1}+\alpha_{2})}


x_{3}= \dfrac{\alpha_{3}I}{p_{3}(\alpha_{1}+ \alpha_{2}+\alpha_{3}}



\end{eqnarray}

pero observamos que $x_{1}^{*}$ y $x_{2}^{*}$ están en función de $x_{3}^{*} $ que acabamos de hallar por lo tanto debemos reemplazar la última en las  dos primeras y hallaremos las cantidades óptimas de los 3 bienes, quedará como ejercicio realizar las operaciones algebraicas  $x_{1}$, por lo tanto teniendo en cuenta que $\alpha_{1}+\alpha_{2}+\alpha_{3}=h$ (por espacio) para $x_{2}$ esto es;


\begin{eqnarray}


x_{2}= \dfrac{  (I-\dfrac{\alpha_{3}I}{h})}{p_{2}(\dfrac{\alpha_{2} + \alpha_{1}}{\alpha_{2}})}

x_{2} = \dfrac{ I \alpha_{2}(1-\dfrac{\alpha_{3}}{h})}{p_{2}( \alpha_{2}+\alpha_{1})}

x_{2}=\dfrac{\alpha_{2}I}{p_{2}(\alpha_{1}+\alpha_{2}+\alpha_{3})}


\end{eqnarray}

así podemos resumir que para los 3 bienes dada la función de utilidad de cobb douglas; 

$$x_{i}^{*}=\dfrac{\alpha_{i}I}{p_{i}(\alpha_{1}+\alpha_{2}+\alpha_{3}}$$

también será bastante sencillo demostrar que el gasto es igual al ingreso.


$$ \sum(p_{i}x_{i}^{*})=\sum p_{i}(\dfrac{I \alpha_{i}}{p_{i}\sum\alpha_{i} }) = I \dfrac{\sum \alpha_{i}}{\sum \alpha_{i}}=I $$


tenga en cuenta este resultado que es muy importante en la teoría del equilibrio.

Ejercicio propuesto; 

maximice $ \sum_{1}^{n} \ln x_{i}^{\alpha_{i}}$ sujeto a $\sum_{1}^{n} p_{i}x_{i}$


\subsection*{Análisis Microeconómico}

En esta sección se exponen brevemente algunos temas relevantes del análisis microeconómico, algunos conceptos básicos que se deberían manejar con presteza  y afianzamiento.

Dado en que en la sección(...) se dio una pequeña introducción a la optimización no restringida y para el multiplicador de lagrange,  el tratamiento de los conceptos básicos en microeconomia requieren un buen manejo de esta sección, pues constituye un requisito.

\subsubsection*{Comportamiento Maximizador}

En el análisis microeconomico se supone que los agentes se comportan como optimizadores de sus recursos así el beneficio $\pi$ que reciben las empresas aunque no es una condición necesaria que sea el fin único, para este caso.


$$\pi(Z)=I(Z)-W(Z)$$

Los beneficios son iguales a los ingresos menos los costos.

así que;

$$\dfrac{ d \pi(Z)}{dZ} = \dfrac{dI(Z)}{dZ}- \dfrac{dW(Z)}{dZ}$$

Observe que los ingresos $I(Z)$ son iguales a $p(Z)Z$ esto es el precio por el producto, observe que a la ves este precio esta en función  de la cantidad de bienes ofrecidos en la economía, cuyo comportamiento se espera este determinado por los patrones de la \textit{ley de oferta y demanda}.

$\dfrac{dI(Z)}{dZ}$ se conoce como el \textbf{Ingreso Marginal $(IM)$} Y $\dfrac{dW(Z)}{dZ}$ se conoce como el \textbf{Costo Marginal $(CM)$} así que;


$$IM = \dfrac{dp(Z)Z}{dZ} = \dfrac{dp(Z)}{dZ}Z + \dfrac{dZ}{dZ}p(Z)= \dfrac{dp(Z)}{dz}Z+p(Z)$$


Ahora para simplificar nuestro análisis(solo por el momento) tendremos en cuenta que los agentes no tienen ninguna influencia sobre los precios del mercado, lo cual significa que $\dfrac{dp(Z)}{dz}=0$
por lo tanto $IM=p(Z)$ así que en el punto donde se maximizan los recursos; $$\dfrac{d\pi(Z)}{dZ}=0$$ 

$$0= IM-CM$$

$$IM=CM$$

para el caso concreto de la competencia perfecta (no se puede influir sobre los precios de mercado) el punto o la cantidad de producto $(Z)$ donde se maximizan los beneficios es cuando el $CM$ iguala al precio de mercado, esto significa que la producción del ultimo producto le cuesta a la firma producirlo exactamente lo que recibe por ella en el mercado. 

Mas adelante analizaremos que pasa cuando el $IM$ no es igual a precio de mercado para esto veremos el concepto de elasticidad de demanda.


La elasticidad de demanda es el cambio en las cantidades de producto adquiridas con respecto a los cambios en los precios, esto en términos porcentuales es decir;


$$\dfrac{dlnZ}{dlnp}=\eta$$

Se asume que $\eta$ es negativa como se enunció anteriormente. Un incremento de los precios en $ 1\%$  disminuye la demanda en $\eta \%.$

Aceptaremos que la función tiene inversa, por el teorema de la función inversa.

$$\dfrac{dlnp}{dlnz}=\dfrac{1}{\eta}$$

Ahora reescribiremos el ingreso marginal, para utilizar el concepto de demanda que será de elasticidad en términos prácticos suponga que los precios incrementan, por lo tanto se esperaría tener mayor ingreso por cada unidad vendida. Sin embargo,  no es del todo cierta esta afirmación depende del grado de sustitubilidad de nuestro producto entre otros factores que nos determinaran el grado de elasticidad de este, es decir;


$$IM= \dfrac{dp(Z)}{dz}Z+p(Z)= p(Z)\left(\dfrac{dp(Z)}{dZ}\dfrac{Z}{p(Z)}+1\right)$$


Utilizando el concepto de elasticidad diremos que;

$$IM=p(Z)\left(1-\dfrac{1}{\abs{\eta}}\right)$$

cuando $\eta$ tiende a infinito significa que nuestro ingreso marginal debe ser igual al precio de mercado, en el caso contrario cuando la elasticidad tiene de cero tendremos un ingreso marginal inferior al precio de mercado.

para ilustrar la importancia en la relación supongamos que $\eta=1$ esto implica que un aumento de los precios en $ 1 \% $ reducirá las cantidades vendidas en la misma proporción quiere decir que nuestro ingreso adicional será nulo, puesto que la perdida de las ventas serán compensadas con los incrementos en el precio.

Una importante implicación empírica de la ecuación anterior es que;

$$ \dfrac{IM}{p}=\left(1-\dfrac{1}{\abs{\eta}}\right)$$

ahora si tenemos en cuenta que en una situación de competencia perfecta $IM=CM$


$$\dfrac{IM-p}{p}=\dfrac{-1}{\abs{\eta}} = \dfrac{p-CM}{p}=\dfrac{1}{\abs{\eta}}$$

como se observa la capacidad de establecer un precio por encima del costo marginal es inversamente proporcional a la elasticidad de demanda de dicho bien, este se denomina como índice de lerner y es utilizado para medir el poder de monopolio de una empresa. pues cada vez que la elasticidad tiende a infinito el costo marginal tiende al precio de mercado y por lo tanto no existe poder de monopolio.

es también posible ver que el precio optimo depende del costo marginal y la elasticidad de demanda.




La importancia de la elasticidad como concepto también se puede ver en la forma o discriminar los precios.

para una empresa que vende $(n)$ bienes tenemos que sus beneficios suponiendo un costo igual para ambos;

$$\pi(Z)=\sum_{i}^{n}I(z_{i}) + CT(Z)$$
$$\dfrac{\partial \pi(Z)}{\partial z_{i}}=IM_{i}+CM=0$$
$$IM_{i}=IM_{i+1}=....=IM_{i+k}=CM$$

Los ingresos marginales obtenidos por cada uno de los productos en los $(n)$ mercados deben ser iguales puesto que si no fuera cierto podríamos vender una cantidad más en cualquier de los mercados más atractivos dado que el costo marginal es igual para cada uno.




$$IM_{i}=p_{i}\left(1-\dfrac{1}{\abs{\eta_{i}}}\right)= p_{i+1}\left(1-\dfrac{1}{\abs{\eta_{i+1}}}\right) =  ... = p_{i+k}\left(1-\dfrac{1}{\abs{\eta_{i+k}}}\right) = IM_{i+k} $$







Así debemos definir para todos los $(n)$ artículos 


supongamos en el caso particular que tenemos dos mercados para brindar el mismo producto, pero el grado de sustitubilidad difiere entre ellos, por lo tanto sus elasticidades también así tenemos para el primer mercado $\eta_{1}$ y para el segundo mercado $\eta_{2}$;




$$p_{1}\left(1-\dfrac{1}{\abs{\eta_{1}}}\right) =p_{2}\left(1-\dfrac{1}{\abs{\eta_{2}}}\right)$$

observe que los precios 

$$ \dfrac{p_{1}}{p_{2}} = \left( \dfrac {1-\dfrac {1} {\abs {\eta_{2}} }} { 1 -\dfrac {1} { \abs { \eta_{1}}}}\right)   $$

si $\eta_{1} > \eta_{2}$ entonces la razón de los precios $ \dfrac{p_{1}}{p_{2}}< 1$ lo cual implica que $p_{1}<p_{2}$  y para el caso contrario cuando cuando $\eta_{1} < \eta_{2}$ entonces $\dfrac{p_{1}}{p_{2}}>1 $ $p_{1}>p_{2}$ por lo tanto a mayor elasticidad menor precio lo cual esta acorde con la capacidad es consistente con el indice de Lerner.

\\

Otro concepto importante en el análisis económico y en especial en lo que concierne al análisis empírico es la elasticidad del ingreso, que permite medir la sensibilidad de los cambios en las cantidades demandadas con respecto a los cambios en los ingresos, en el modelo anterior  del multiplicador lagrangeano  aceptamos que se maximizaba la utilidad gastando todo el ingreso, esto implica que el gasto también lo hará en la misma medida; un incremento en el ingreso de $\iota$ \% también incrementará el consumo en esa misma cantidad $dG/dI=1$;

$$1 = \sum p_{i}(\dfrac{dx_{i}}{dI})$$
si multiplicamos cada término de la sumatoria por $\dfrac{I}{I}$ y $\dfrac{x_{i}}{x_{i}}$ y se organizan los términos de tal manera que $\dfrac{p_{i}x_{i}}{I}$


$$ \sum \dfrac{p_{i}x_{i}}{I} \dfrac{dx_{i}}{dI} \dfrac{I}{x_{i}}$$

obtenemos la proporción del ingreso  gastado en el bien $x_{i}$ denominado $\omega_{i}$ y la elasticidad ingreso $ \varepsilon_{i}$ de tal bien en cuestión esto quiere decir que 

$$ \sum \omega_{i} \varepsilon_{i}=1 $$  que deben existir bienes inferiores,  normales y superiores; cuya demanda aumenta cuando aumenta el ingreso, disminuye o se mantienen constantes.


En la teoría de la firma como vimos anteriormente el concepto de Ingreso marginal y costo marginal son importantes para obtener el punto donde nuestro productor maximiza su ganancia.  El costo marginal se definió como los cambios en los costos totales debido a los incrementos en la producción, existe una relación ente la marginalidad y el promedio como se verá el siguiente análisis también se aplicó al ingreso marginal;

el costo total es equivalente al costo unitario o promedio $w$ por la cantidad total de producción( se comporta siempre así el costo promedio revisar en Krugman para comparar el otro supuesto); 

$$\dfrac{dW(Z)}{dZ}= \dfrac{w(Z).Z}{dZ}$$


\subsubsection{la firma}



Ya hemos hablado de algunos conceptos microeconomicos abstractos como la elasticidad, y la marginalidad del ingreso y los costos que son estructuras adyacentes a situaciones diarias, ahora entremos un poco mas al campo del análisis de la firma, advirtiendo  de antemano que el análisis del consumidor no cambia absolutamente en nada sustancial (son prácticamente lo mismo en términos del análisis), por lo tanto será muy fácil para el lector introducirse en este cuando se ha abordado el análisis de la empresa.

dada una función de producción 
$Z=f(x_1,x_2,x_3,...,x_n)$, definimos los cambios en la producción cuando aumentamos la cantidad de un factor productivo, manteniendo a los demás factores constantes, esto es equivalente a la definición de la derivada parcial, expuesta con anterioridad, es decir que el \textbf{producto marginal} $PMx_{i}$ del factor $xi$ es la derivada parcial de la función de producción con respecto al factor $x_{i}$.

$$\frac{\partial Z}{\partial x_{i}}= PMx_{i}$$

dado esto nos podemos preguntar entonces, como cambia la producción total cuando aumentamos los factores productivos, para este caso utilizamos el concepto de diferencial total; 

$$dZ= \dfrac{\partial Z}{ \partial x_{i}} dx_{i} + \dfrac{\partial Z}{ \partial x_{i+1}} dx_{i+1} +...+\dfrac{\partial Z}{ \partial x_{n-1}} dx_{n-1} +\dfrac{\partial Z}{ \partial x_{n}} dx_{n}$$

que lo podemos reescribir de la siguiente manera;

$$dZ=\sum_{1}^{n}=PMx_{i}dx_{i}$$

sin embargo los planes de producción requieren siempre una combinación óptima de los factores productivos, es decir cual sería la combinación optima de insumos para producir una cierta cantidad de producto (Z).   

Ahora es necesario introducir un concepto nuevo,  el de Tasa de Sustitución de Factores productivos,(TSF) en los textos tienen otras denominaciones. Sin embargo, no será necesario adoptar convicciones por ahora pues solo queremos introducir los conceptos o recordar el análisis adyacente, es decir a que cantidad del factor $x_{i+1}$ debo renunciar para obtener la misma producción pero ahora utilizando una cantidad del factor $x_{i}$ adicional.


Es bastante fácil ver que  bastará  con que $dZ=0$ para definir nuestra TSF. Para simplificar tomemos el caso particular cuando $n=2$;


$$\sum PMx_{i}dx_{i}=0$$

por lo tanto haciendo las operaciones algebraicas, tenemos que; 

$$  TSF= (-1) \dfrac{PMx_{i}}{PMx_{i+1}}=\dfrac{dx_{i+1}}{dx_{i}}$$

supongamos que para un nivel determinado de producción, el producto marginal del factor $x_{i}$ es el doble del del factor $x_{i+1}$ si quisiéramos mantener constante la producción, debemos  sustituir los factores, pues a menos que se especifique otra cosa ambos productos marginales son positivos. se necesita la sustitución por que es más barato por ejemplo, entonces nuestra tecnología de producción nos indicaría que debemos reducir el uso de dos unidades del factor $x_{i+1}$ para aumentar en una unidad del uso de $x_{i}$.

el producto marginal está relacionado también con un concepto  que usualmente es útil en el análisis empírico; el de producto promedio del factor $x_{i}$.

$$PMDx_{i}= \dfrac{Z}{x_{i}}$$

$$\dfrac{\partial PMDx_{i}}{\partial x_{i}}=0$$

$$\dfrac{\partial PMDx_{i}}{\partial x_{i}} = \dfrac{PMx_{i}x_{i}-Z}{x_{i}^{2}} = x_{i}(PM_{i}-PMDx_{i})=0 \therefore PMx_{i}=PMDx_{i}$$ 

esto quiere decir que cuando el producto promedio es el máximo, en ese punto es igual al producto marginal.

decíamos anteriormente que $I= p(Z).Z$, en efecto cuando los mercados son competitivos podríamos suponer que $IM=p$ ahora pensemos en el caso particular donde $I=p(Z).f(x_1,x_2,x_3,...,x_n)$ 

simplifiquemos y supongamos que el precio es de mercado, 

$$I=p.f(x_1,x_2,x_3,...,x_n)$$
$$\dfrac{\partial I}{\partial x_{i}}= p(PMx_{i})$$





\vspace*{0,9cm}

\subsection{Crecimiento económico}


\subsubsection{La función neoclásica}
La función de producción Cobb-Douglas es simplemente el caso especial cuando $i=2$ de la siguiente función;

$$\prod_{i=1}^n x_i^{\alpha_i}  $$ [1]

Para cualquier combinación de insumos, podemos demostrar que [1] presenta rendimientos constantes a escala (si se quiere) siempre y cuando 

 $$ \lambda^{\sum_{i=1}^n\alpha{_i}}=\lambda $$
 
 
puesto que; 



$$\prod_{i=1}^n \left(\lambda x_i \right)^{\alpha_i}=\lambda^{{\sum_{i=1}^n\alpha_i}}\prod_{i=1}^n x_i^{\alpha_i}$$
 
 
 
 
 En otras palabras que la suma de las participaciones de los factores productivos en el ingreso sea la unidad $\sum_{i=1}^n\alpha{_i}=1$
 
 al suponer esto estaríamos condicionando a que la economía es lo bastante grande para acabar con los rendimientos crecientes de escala Romer(xxx).
\\ 
 
 por lo tanto si se observa el punto, para determinar el grado de homogeneidad, es suficiente sumar los coeficientes de participación.
para el caso concreto cuando los exponentes son $\alpha$ y $1-\alpha$ sin importar el valor $\alpha$ será de grado 1. 

Cuando la función de producción es homogénea de grado $\phi$, podemos considerarla de manera intensiva esto quiere decir que si

 
$$Z=f(\lambda x_1,\lambda x_2,\lambda x_3,...,\lambda x_n)\\ =
\lambda^{\phi}f(x_1,x_2,x_3,...,x_n) $$

así que queremos expresarla de manera intensiva diremos que $\lambda=\frac{1}{x_{i}}$
y obtenemos que;


$$Z=f( \dfrac{x_1}{x_i},\dfrac{x_2}{x_i},\dfrac{x_3}{x_i},..1..,\dfrac{x_n}{x_i}) =
\dfrac{1}{x_i^{\phi}}f(x_1,x_2,x_3,...,x_n) $$

observe que el 1 de la función quiere decir que hemos escogido un insumo arbitrario. 

$Z = f(x_1,x_2,x_3,...,x_n) =  x_i^\phi f( \dfrac{x_1}{x_i},\dfrac{x_2}{x_i},\dfrac{x_3}{x_i},..1..,\dfrac{x_n}{x_i})$

Observe que en los modelos requerimos por lo general que $\phi=1$ homogenea de grado uno esto implica que 

$$
\dfrac{Z}{x_i}=f( \dfrac{x_1}{x_i},\dfrac{x_2}{x_i},\dfrac{x_3}{x_i},..1..,\dfrac{x_n}{x_i})$$

para el caso concreto de [1] podemos seleccionar cualquier insumo $x_j$ y reescribirla como; 


$$Z=\prod_{i=1}^n \left( \frac{x_i}{x_j}\right)^{\alpha_i} x_j^{\phi}$$






Es útil que el factor que hemos escogido sea la población(cuando decimos que esta es igual a la fuerza laboral) puesto que esto nos determinará una función de producción per cápita, así


 $$z=\prod_{i=1}^n \left( \frac{x_i}{x_j}\right)^{\alpha_i}  =\prod_{i=1}^n k_{i}^{\alpha_i}$$  [2]

donde $z$ es el producto per cápita y $k_{i}$ es el argumento o factor productivo per cápita.

Lo importante a notar ahora de nuestra función de producción per capita es que nos permite estudiar el crecimiento del producto por persona, sin embargo hay otras condiciones que se deben cumplir como las condiciones de INADA.

si logramos establecer esta relación de nuestra función de producción [1] tal que el producto marginal del factor $x_{i}$ sea positivo pero decreciente estaremos ante una satisfactoria función de producción denominada neoclásica.


El producto Marginal; 

$$PMx_i=(x_i)^{-1}\alpha_i\prod_{i=1}^nx_i^{\alpha_i}$$

Es positivo para todos las cantidades positivas de argumentos(insumos). Sea cual sea el valor de la participación $\alpha_i$ del factor sera positiva mientras esta también lo sea, esto quiere decir que no será cierta nuestra condición si y solo $\alpha_{i}=0$. lo que nos daría una productividad marginal nula.


El producto Marginal decreciente;

$$\frac{\partial PMx_i}{\partial x_i}=(x_i)^{-2}\alpha_i(\alpha - 1)\prod_{i=1}^nx_i^{\alpha_i}$$
es negativo para cualquier cantidad positiva de los argumentos, recuerde que se está hablando de una derivada parcial, lo cual implica los rendimientos de un insumo cuando los demás permanecen constantes. Ahora observe que nuestra condición no será cierta si y solo si $\alpha_i \geq 1$, lo cual sería un supuesto poco realista puesto que no se daría la famosa premisa de \textit{los rendimientos decrecientes}, seguiríamos obteniendo producto indiscriminadamente o no reduciríamos la producción cuando los factores sobrepasen su eficiencia.





Los supuestos anteriores de productividad se derivan de las condiciones de \textit{inada}

$$\lim_{x_i\longrightarrow \infty}PMx_i=0$$
$$\lim_{x_i\longrightarrow 0}PMx_i=\infty$$

\newpage
las dos propiedades se pueden comprobar de ;
$$\lim_{x_i\longrightarrow \xi}\frac{1}{x}$$ donde $$ {\xi=\infty, 0}$$.



Observemos que de manera inmediata se puede deducir que la misma estructura de se refleja en la función intensiva como se comporta con respecto [2] con respecto a sus productividades.



$$PMk_i=(k_i)^{-1}\alpha_i\prod_{i=1}^nk_i^{\alpha_i}$$




$$\frac{\partial PMk_i}{\partial k_i}=(x_i)^{-2}\alpha_i(\alpha - 1)\prod_{i=1}^nk_i^{\alpha_i}$$

\section{Modelos de crecimiento}
En esta sección se intenta exponer como diferentes formas funcionales pueden ayudarnos a entender el crecimiento económico.

lo importante de nuestro modelo es determinar como evoluciona la economía en el tiempo, cuales son los cambios en el producto de la economía, para eso resulta el producto por habitante una buena medida de análisis, por ejemplo Mozambique en 1990 era el país mas pobre del mundo en términos de PIB per cápita, en un cuarto de siglo, con tasas de crecimiento promedio para el periodo (1990-2015) logró dejar de ser el país mas pobre y logró tener un PIB per cápita casi 5 veces mas grande.





Como se evidenció en el párrafo anterior, usualmente la evolución del producto per capita, es una medida del crecimiento de las economías.

Solow(xxx) se interesó en el aporte que tiene el capital físico, por eso debemos preguntarnos como es su evolución en el tiempo y como afecta el producto.



$$\dfrac{K}{L} =k $$

$$ \dfrac{d(\dfrac{K}{L})}{dt}= \dfrac {dk}{dt}$$


$$ \dfrac{dk}{dt}=\dfrac{\dfrac{dK}{dt}.L-\dfrac{dL}{dt}K}{L^{2}} = \dfrac{\dfrac{dK}{dt}}{L}-\dfrac{\dfrac{dL}{dt}}{L}k$$

[3]

esta expresión se obtiene aplicando la regla del cociente expuesta en la sección(...).





\

Supondremos también que $(L)$ crece a una tasa constante con respecto al tiempo $n$.  Recuerde en la sección(....) se demostró que $\dfrac{d Ln h(x)}{dx}=\dfrac{dh(x)}{h(x)}$.

\

por lo que [3] finalmente se puede escribir como; 

$$\dfrac{dk}{dt}=\dfrac{\dfrac{dK}{dt}}{L}-nk$$ [3.1] 

que será una ecuación fundamental en el desarrollo de lo aquí expuesto.






Tengamos en cuenta que el $Z$ es nuestro producto.

$$Z=I+C$$

$I,C$ son la inversión y el consumo respectivamente.

Ahora podemos desagregar la inversión en dos componentes, la depreciación y la inversión neta.

$$I=K^{*}+\delta K$$

de entrada se supone que, esta tasa de depreciación $\delta$ es constante. Además, el consumo no es otra cosa que la parte que no  se ahorra del producto; $(1-s)Z$.

así se obtiene;

$$Z= (1-\psi)Z+K^{*}+\delta K$$
$$Z=Z-\psi Z +K^{*}+\delta K$$
$$\dfrac{dK}{dt}=K^{*}= \psi Z - \delta K $$ [4]

El capital neto, es la parte del ahorro  que no se ha contabilizado en la depreciación, esta última ecuación también será fundamental.

así obtenemos de [3.1] y [4];


$$\dfrac{dk}{dt}=\dfrac{K^{*}}{L}-nk$$

$$\dfrac{dk}{dt}=\dfrac{\psi Z - \delta K}{L}-nk$$

$$\dfrac{dk}{dt}=\dfrac{\psi Z}{L}- \delta k -nk = \dfrac{\psi Z}{L}-(\delta  + n)k $$

[5]

Lo interesante hasta  momento es que el modelo en general nos permite conocer, sin especificar una formula funcional para el producto $Z$ que el crecimiento del capital percapita es la diferencia entre el ahorro de la producción por trabajador y el descuento de la depreciación y la tasa de crecimiento de la población. 




Ahora nos falta definir $Z$, si decimos que Z toma la forma funcional de [1] podríamos escribir [5] de la siguiente manera.

$$ \dfrac{ \psi \prod_{i=1}^nx_i^{\alpha_i} }{L}- (\delta + n)k$$

ahora, decimos que existe un $x_{j}=L$, entonces escribiremos [1] de manera intensiva esto es;


$$Z=\prod_{i=1}^{n}\left(\dfrac{x_i}{L}\right)^{\alpha_i}L$$

de lo anterior se deduce que hemos supuesto $\phi=1$ o lo que es lo mismo que $\sum_{i=1}^{n} \alpha_i =1$ (nuestra función presenta rendimientos constantes a escala).

para calcular la función de producción por trabajador $z=Z/L$ basta expresar; 

$$z=\prod_{i=1}^{n}\left(\dfrac{x_i}{L}\right)^{\alpha_i}$$


podríamos ahora considerar la tasa de crecimiento del capital per capita, como una función del producto percapita.

$$
\dfrac{dk}{dt}= \psi \prod_{i=1}^{n}\left(\dfrac{x_i}{L}\right)^{\alpha_i} -(\delta + n)k$$ 


Recuerde que en [2] los insumos o argumentos intensivos los definimos como $k_{i}= \left( \dfrac {x_{i}}{x_{j}} \right)$ y que  existe entre todos los factores uno denominado capital intensivo y es el que nos interesa en el tiempo,por lo tanto, la tasa de crecimiento del capital por trabajador $(\Omega k)$ es; 

$$  \Omega k=\dfrac{\dfrac{dk}{dt}}{k}=  k ^ {- 1 } \psi \prod_{i=1}^{n}\left(\dfrac{x_i}{L}\right)^{\alpha_i} -(\delta + n)k $$



Un caso particular cuando $i=3$, $ Y=AK^{\alpha}L^{1-\alpha}E^\tau$ ahora escogemos nuestra función de manera intensiva, seleccionando para este caso los trabajadores $L$ como lo ya hicimos anteriormente;

de lo cual 

$$Y=A\left(\dfrac{K}{L}\right)^{\alpha}\left(\dfrac{E}{L}\right)^{\tau}L^{\tau + \alpha + \ 1- \alpha}=Ak^\alpha \left(\dfrac{E}{L}\right)^{\tau} L^{\tau + 1}$$

Observe que nuestra función es homogénea de grado $\tau + 1 $, se ha sugerido que $E=k$ o $E=K$, probemos el primer caso.



$$Y=Ak^{\alpha}\left(\dfrac{k^{\tau}}{L^{\tau}} \right) L^{\tau +1}$$

$$ Y= Ak^{\alpha}k^{\tau}L = Ak^{\alpha + \tau}L$$

entonces la producción por trabajador es simplemente $y=Ak^{\alpha + \tau}$.


para el caso donde $E=K$ debemos tener en cuenta que en lo posible necesitamos establecer una relación intensiva o per capita por lo que $K=kL$ de ahí se obtiene que;

$$Y=Ak^{\alpha}\left(\dfrac{kL}{L}\right)^{\tau}L^{\tau + 1} = Ak^{\alpha + \tau } L^{\tau +1 }$$

Entonces la producción por trabajador es $y= Ak^{\alpha + \tau }L^{ \tau}$. La diferencia fundamental, cuando la externalidad se entiende de manera agregada  es que la producción por trabajador en esta última   obtiene mayores niveles de producto a mayor población trabajando, con respecto a la ecuación per capita teniendo en cuenta que la externalidad es de capital per capita. 



Podemos entonces estudiar el comportamiento del capital per cápita en el tiempo  con cualquiera  de estas funciones utilizando [5].



Utilizando la función de producción cuando la externalidad es igual al capital por persona ( $E=k$ ) obtenemos que; 

$$
\dfrac{dk}{dt}= \psi Ak^{\alpha + \tau}-(\delta + n ) k$$



$$\Omega k= \psi A k^{\alpha+r-1}-(\delta + n)$$

Para analizar el comportamiento de la tasa de crecimiento con respecto al capital debemos entender los tres casos factibles. $i) \tau + \alpha =1, ii) \tau + \alpha < 1 $, y $iii) \tau + \alpha > 1$.

\
\
       
\textit{primer caso $i)$}

\


Como se observa la tasa de crecimiento cuando $\tau + \alpha =1$ es  una tasa constante de crecimiento, es como si lo función de producción per cápita hubiese tomado la forma de $y=Ak$.

$$\Omega K = \psi A - (\delta + n)$$ [6]




Note que el crecimiento no es más que la diferencia entre una tasa de la productividad causada por la tecnología menos la depreciación del capital y la tasa de crecimiento poblacional, este se conoce como un modelo de crecimiento endógeno.


la siguiente figura ilustra el comportamiento de esta tasa de crecimiento:



el siguiente modelo presenta la ley de los rendimientos marginales decrecientes, lo cual implica el caso $ii)$ a medida que aumenta el capital disminuye la parte ahorrada hasta converger a un estado de equilibrio donde 
$\dfrac{\psi A}{k^{1 - \alpha - \tau}}=(\delta + n)$







el caso $iii)$ es inestable como veremos más adelante puesto que toma la forma producto positivo debido  a que presenta rendimientos no decrecientes dado que $\alpha + \tau  -1 > 0$






\definecolor{xdxdff}{rgb}{0.6588235294117647,0.6588235294117647,0.6588235294117647}
\definecolor{uuuuuu}{rgb}{0.26666666666666666,0.26666666666666666,0.26666666666666666}
\definecolor{wwwwww}{rgb}{0.4,0.4,0.4}
\definecolor{qqwuqq}{rgb}{0.12941176470588237,0.12941176470588237,0.12941176470588237}


\begin{tikzpicture}[line cap=round,line join=round,>=triangle 45,x=0.8cm,y=0.8cm]
\draw[->,color=black] (0.,0.) -- (16.,0.);
\foreach \x in {,1.,2.,3.,4.,5.,6.,7.,8.,9.,10.,11.,12.,13.,14.,15.}
\draw[shift={(\x,0)},color=black] (0pt,-2pt);
\draw[->,color=black] (0.,-0.01) -- (0.,9.);
\foreach \y in {,1.,2.,3.,4.,5.,6.,7.,8.}
\draw[shift={(0,\y)},color=black] (2pt,0pt) -- (-2pt,0pt);
\clip(0.,-0.01) rectangle (16.,9.);
\draw[line width=1.2pt,color=qqwuqq,smooth,samples=100,domain=0.0:16.0] plot(\x,{30.0/(\x)});
\draw[line width=1.2pt,color=wwwwww,smooth,samples=100,domain=0.0:16.0] plot(\x,{3.0});
\draw [dash pattern=on 4pt off 4pt] (10.,3.)-- (10.,0.);
\draw (10.16,3.68) node[anchor=north west] {$y^{\bullet}$};
\draw (10.16,0.68) node[anchor=north west] {$k^{\bullet}$};
\draw (0.1,3.92) node[anchor=north west] {$(\delta+n)$};
\begin{scriptsize}
\draw[color=wwwwww] (-3.24,2.87) node {$g$};
\draw [fill=uuuuuu] (10.,3.) circle (1.5pt);
\draw [fill=xdxdff] (10.,0.) circle (2.5pt);
\end{scriptsize}
\end{tikzpicture}


Analicemos ahora el caso particular cuando la externalidad toma el valor del capital agregado;


$$\dfrac{dk}{dt}= \psi Ak^{\alpha + \tau}L^{\tau} - (\delta + n)k$$

$$ \Omega k=  \psi Ak^{\alpha + \tau -1}L^{\tau} - (\delta + n)$$

Esta tasa de crecimiento del capital se parece mucho a la anterior la diferencia aquí radica es que cuando se comporta como una función $Ak$ lo cual implica que $\alpha + \tau -1  = 0$ las diferencias se deben a la población pero como la población esta creciendo a una tasa constante tendremos rendimientos exponenciales por lo tanto para simplificar el modelo se asume que $n = 0$. Así se dice que las diferencias en población representará una  mayor tasa de crecimiento del capital por persona en el tiempo  y la única forma de reducirla es ampliando la constante de depreciación $\delta$.

Hallar el capital de estado estacionario, requiere que $\Omega k=0$ o que$\dfrac{dk}{dt}=0$, siendo;


$$ 0 = \dfrac{dk}{dt}= \psi Ak^{\alpha + \tau}L^{\tau} - (\delta + n)k$$

$$(\delta + n)k =\psi Ak^{\alpha + \tau}L^{\tau} $$

$$ \left(\dfrac{\delta + n}{\psi AL^{\tau}} \right)= k^{\alpha + \tau -1}$$


por lo tanto el capital de estado estacionario está en función de los parámetros;

$$k^{\bullet} = \left( \dfrac{\psi AL^{\tau}}{\delta +n} \right)^{\dfrac{1}{1-\alpha - \tau} }$$
 
 si se desea conocer la tasa de crecimiento de cualquier variable sea agregada o per capita solo basta recordar su definición por ejemplo observemos de este modelo que $Y/L=y$ pero como en el estado estacionario la tasa de crecimiento del producto per cápita es    nula entonces la derivada de las funciones logarítmicas en el tiempo nos darán las tasas de crecimiento respectivas;
 $$LnY=lny + Ln n$$ de lo cual se puede inferir que el producto no crecerá  en términos agregados si tampoco lo hace la población.  El caso contrario se presenta en el modelo $Ak$ como el producto per cápita crece a una constante igualmente lo hace el producto así;
 
$$\Omega k= \psi A  -(\delta+n)$$


por lo tanto el producto agregado crecéra a una tasa de 

$$\Omega Y= \psi A  -(\delta+n) + n= \psi A -\delta$$.


por último podríamos encontrar el estado estacionario para la forma funcional de [1] cuando es homogénea de grado 1

$$ \dfrac{dk}{dt} = \psi  \prod_{i=1}^{n}\left(\dfrac{x_{i}}{x_{j}}\right)^{\alpha_{i}} - (\delta + n)k$$


$$ \dfrac{dk}{dt} = \psi  \prod_{i=1}^{n}\left(\dfrac{x_{i+1}}{x_{j}}\right)^{\alpha_{i+1}} k^{\alpha_{i}} - (\delta + n)k$$

ahora $k^{\alpha_{i}}$ es fijo. es nuestro factor productivo capital per capita.



siendo el capital de estado estacionario;

$$k^{\bullet}=\left(\dfrac{\psi  \prod_{i=1}^{n}\left(\dfrac{x_{i+1}}{x_{j}}\right)^{\alpha_{i+1}}}{\delta +n}\right)^{\dfrac{1}{1-\alpha_{i}}$$

sería bueno preguntarnos de manera general como cambia el nivel de estado estacionario con los cambios en los parámetros.

\subsection{Teorema de Euler}

Existe un teorema muy importante utilizado con frecuencia en economía denominado como el teorema de Euler y es útil para el análisis microeconómico.


lo primero que debemos entender es que si $Z=F(x,y)$ es homogenea de grado 1, entonces;


$$Z=\dfrac{\partial Z}{\partial x}x + \dfrac{\partial Z}{\partial y}y$$



\begin{theorem}

Si $Z$ es una función de n insumos y homogénea de grado $\phi$, entonces $\dfrac{dZ}{d \lambda}=\phi \lambda ^{\phi -1}Z = \sum_{i=1}^{n} \dfrac{\partial f(x_1,x_2,x_3,...,x_n)}{\partial x_{i}}x_{i}$


\end{theorem}

se puede llegar a esa igualdad de la siguiente manera;

$$Z=f(\lambda x_1,\lambda x_2,\lambda x_3,...,\lambda x_n)\\ =
\lambda^{\phi}f(x_1,x_2,x_3,...,x_n) $$


$$\dfrac{\partial f(\lambda x_1,\lambda x_2,\lambda x_3,...,\lambda x_n)}{\partial \lambda}=\dfrac{\partial f(\lambda x_1,\lambda x_2,\lambda x_3,...,\lambda x_n)}{\partial \lambda x_{i}} \dfrac{\partial \lambda x_{i}}{\partial \lambda}=\dfrac{\partial f(\lambda x_1,\lambda x_2,\lambda x_3,...,\lambda x_n)}{\partial \lambda x_{i}} x_{i} $$ [6]

esto se obtiene  del lado derecho como es una igualad debemos derivar al lado izquierdo ,así ;

$$ \dfrac{ \partial \lambda^{\phi}f(x_1,x_2,x_3,...,x_n)}{\partial \lambda} = \phi \lambda ^{\phi -1 }f(x_1,x_2,x_3,...,x_n)$$ [7]

igualando [6] y [7] obtenemos;


$$ \dfrac{\partial f(\lambda x_1,\lambda x_2,\lambda x_3,...,\lambda x_n)}{\partial \lambda x_{i}} x_{i} = \phi \lambda ^{\phi -1 }f(x_1,x_2,x_3,...,x_n)$$

por lo tanto para el diferencial total tenemos que suponiendo que $\lambda=1$

$$\dfrac{dZ}{d \lambda}= \phi \lambda ^{\phi -1 }f(x_1,x_2,x_3,...,x_n)= \sum_{i=1}^{n}  \dfrac{\partial f(x_1,x_2,x_3,...,x_n)}{\partial x_{i}}x_{i}   $$

Suponiendo que la función es homogénea de grado uno (1) 


$$f(x_1,x_2,x_3,...,x_n)= \sum_{i=1}^{n}  \dfrac{\partial f(x_1,x_2,x_3,...,x_n)}{\partial x_{i}}x_{i}   $$


para dos variables tenemos el resultado enunciado al principio de la sección; 

$$Z = \dfrac{f(x_{1},x_{2})}{\partial x_{1}}x_{1} + \dfrac{f(x_{1},x_{2})}{\partial x_{2}}x_{2}$$

como se quería comprobar.



\section{integral}

El cálculo integral es complementario en el diferencial y también se utiliza en el análisis económico. este método


las integrales tienen una relación potencial con 





sin embargo si se observa una esfera es una circunferencia revolucionada, esto quiere decir que en cierta parte hay pequeños cilindros diferenciales.
lo cual implica que hay que sumar las áreas de los cilindros.

entonces vemos que una figura girada en un plano se presenta un solido de revolución, así podremos obtener muchas más aplicaciones de esta;

la formula obtenida para el volumen de una esfera es
$$ V=4/3 \pi r^{3}$$.

en el estudio los libros de euclides ya aparecen el cálculo de el volumen de algunos sólidos[M] [ML] sin embargo se utilizaba el método de exhausición, por que no se tenía el calculo integral.

pensemos en lo siguiente al revolucionar la circunferencia,(piense que hemos revolucionado en el eje x) con ese mismo radio estamos generando un cono lo cual implica que el radio debe establecerse entre los puntos x y y lo cual establecemos  con el \emph{teorema de pitágoras} para calcular la distancia entre dos puntos, por supuesto que ya hemos utilizado este teorema pero aquí es necesario utilizarlo para fines prácticos así que $r=\sqrt{x^2+y^2}$.

entonces pensemos que el eje x es la altura por lo tanto necesitamos que su altura sea infinitesimal así obtenemos que este cilindro tiene un volumen de 
$$ \pi r^{2} dx$$
observamos que el radio es equivalente al valor del eje $y$ así que $r=y$ y $y=\sqrt{r^2-x^2}$ para al final obtener;

$$\int_{-r}^{r} \pi(r^{2}-x^{2}) dx =4/3 \pi r^{3} $$ así obtenemos el esperado resultado.


los logaritmos también tienen su relación con las integrales, en este proceso es necesario tratar de articular lo más posible el proceso intuitivo de las integrales y su aplicación.


\end{document}
