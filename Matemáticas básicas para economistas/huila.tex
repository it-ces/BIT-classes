\documentclass[12pt,letterpaper]{report}
\usepackage[latin5]{inputenc}
\author{Iván Andrés Trujillo}
\begin{document}


\subsection*{summary }{\textbf{Trujillo 20132120845}}

\
\

The imcome\´s level of a territory can says us, its development in a raw way. However this is not a goal by itself. The really important is understand how get a hihger level of GDP (in the case there is not a measure of income). Altough, There are different determinants about the performance of a territory, even could think about what is the best indicator for having a best controll in the policy and its efficiency, in others words how we can assess the development of the a economy, but is a complicated task. for instance, we can use expenditure in enforcment law, the numbers of cases of corruption, the same gdp´s growht, and so on, the police maker have a set of posibles alternatives what looks like equally efective.



We can search in the literature, although in the literature we can´t find one coclusion difinitive. we have a  toolbox for affect the outcomes in the economy performance, this suggest us that to answer; What is the impact that have other factors (culture, institutions and geographical variables) in our region? is a complicated question, altough necessary. is hard established a mesaure because of how we states above the outcomes can have been affect by multiples factors. 


The modern theory of economic growht have taken in mind the development of institutions how the main factor for determine the success path, but this quality is correlate with the history geographical factors, in the colony process, or are independlenty maybe it´s because a other things how the leadership or ruling parties how in bowstana and singapure \cite{rodriguez}. They win its indepdence barely in 1966 and 1965 respectively. Today have a great performance  economic (in terms of gdp per capita) that other countrys that get its independece two  centuries ago. for example today according to world bank´s data colombia have roughly 11,4\% gdp´s singapore and its almost equal to gdp´s bostwana in a 95,2\% but colombia  in 1829 gets its indepdence almost xxx year beforte that bowstanta did it

according to \cite{institutions} we cant be sure that the level of income of a territory not affect the quality of institutions, in other words that the insitutiosn are endogenous. Which means that the analysis must be corrected, or a linear analysis not it´s adequate.


We must take into account that there are a paradoxical situation, the performance of a economy have a several determinats, we can  conclude that despite that there are many theories and empirical estudies dont exist convergence about a consitency     plan for to know what is better way?.Recenlty it has been studied for the economist in colombia 

%notas de imprimir  uno;

the role that play the geography is secondary, it has a indirect effect, the main role is played by the institutions quality´s \cite{clavijo}, the author state that "engine of growth" where play a role more important the policies than the exogenous factors; how demography and geography, but remeber that the quality of this politics depend on the institutions \cite{rodriguez} exposes that the goverment was composed by foreigns with hihger human capital "Botswana benefited from the presence of experts on public policy design, and the result of it was evident, especially when compared with other regimes that were deliberately against hiring foreigners for the public service (eg.Somalia.)"  pag 29.

this approach take  in mind the predictions of solow´s model, where is more important the calpital-labor rate.He attemped quantify the economic growth. and found that 1950-2002, that rate of investment play a important role.

\cite{mendoza} have a interesting empirical study what focus on the imppact the a geographical variables set in the income per capita.

they found that the institutions have a important role in the gdp discrepances. the empirical studies have had a great importance in the literature because of a relative new evidence support a set of ideas that rely  on in the statistical analisys. 

\cite{sanchez}  focus on examining the direct relationship in colombia municipalities with some geographical variables and income per capita by the period 1973-1995.

according to \cite{sanchez} the geography affects the income per capita and its growth in a avarage about 35\% but there is another important "found" these variables affect more a to poor municipalities than rich ones. Also, the authors concluded too, that the policie and human factor are  much important. \textit{ "at the begining of the 1980s, Bogotá´s economic growth and  development left the rest of the principal Colombian cities behind."} how we have seen it, it´s a common denominator in the currently situation. bogotá was or is the centre of the colombian gravity model.


regarding to the literature in colombia we have found that the geography play  a crucial role, indepently if is direct or indirect. the productivity of land, and its cost of oportunity.




\cite{diaz} have estimated  the relation the efect that have over the studied or human capial over the employment rates in the colombiana municipalities and she has founded that the behavior of the work seems to be of the factors complementaries. this mean that increases in the skill of workers increseas de employmen the non-workers too, this is very insteristing because this positive externaliti  have a effect over other social variables how crime, and goverment assitence.






\bibliographystyle{apalike}


\bibliography{refrenceces}


\end{document}